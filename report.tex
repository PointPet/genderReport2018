\documentclass[12pt]{article}
\usepackage{graphicx}
\usepackage{amsmath}
\usepackage{amssymb}
\usepackage{amstext}
\usepackage{amsfonts}
\usepackage{mathrsfs}
\usepackage{makeidx}
\usepackage{multirow}
\setlength{\parindent}{0pt}
\title{STEM Gender Gap in University Education}
\date{June 2017}
\author{Peter Pointner ,Thomas Sulzbacher, Yang Hua}

\begin{document}
	\pagenumbering{gobble}
	\maketitle
	\newpage
	
	\tableofcontents
	\newpage
	\pagenumbering{arabic}
	\section{Abstract}
In this paper our team focuses on the gender gap which appears in Science Technology Engineering Mathematics and Computer ,henceforth STEM, majors in university education. First of all we want to give an overview of the current situation of inscription rates in different STEM related majors at Johannes Kepler University Linz, from now on  JKU, and how the development progressed from 2002 to 2016 based on \cite{studienwahl_jku} \cite{eq_1} \cite{eq_2} \cite{eq_3}. After taking a close look at the situation in Austria we want to focus on reasons for the decisions made when choosing a university major and what different influence people are committed.
In the next section the gender gap in STEM majors on universities in Germany is evaluated and compared against the situation in Austria. The main focus ,of the second part of this report, is to take a closer look at the international situation of the STEM gender gap. At the end of the report a conclusion of our findings will be presented.    
 	\section{Situation in German speaking countries}
	\subsection{Situation at JKU}
	\subsection{Situation in Germany}

%\begin{figure}[h!]
%	\includegraphics[width=\linewidth]{vq_priciple.jpg}
%	\caption[Vector Quantization Principle Taken from: \cite{vq_2}]{Vector Quantization Principle.}
%	\label{fig:vq}
%\end{figure}
\newpage
\bibliography{gender}
\bibliographystyle{unsrt}
\newpage
\listoffigures
\listoftables
\end{document}