	\section{Leaving Europe}
	In this section we are going to look at the general development throughout the whole world, based on some real-life reports and methods to compare gender affinity between countries.
	
	\subsection{Measuring global development}
	
	\subsubsection{Gender Gap Index}
	
	Where no single measure can capture the complete situation, the Global Gender Gap Index seeks to measure one important aspect of gender equality: The relative gaps between women and men across four key areas:
	\begin{itemize}
		\item health and survival
		\item education
		\item economic participation
		\item political empowerment
	\end{itemize}
	The global Gender Gap Index was introduced in 2006 to track progress. In 2017 144 countries have been analyzed. 
	The ranking are designed to raise global awareness of the challenges posed by gender gaps and the opportunities created by reducing them. \cite{t_gender}
	
	\subsubsection{Quality and Representativity}
	
	So what explains the tendency for nations that have traditionally less gender equality to have more women in science and technology than their gender-progressive counterparts do?
	
	it could have to do with the fact that women in countries with higher gender inequality are simply seeking the clearest possible path to financial freedom. And often, that path leads through stem professions.
	
	The issue doesn’t appear to be girls’ aptitude for stem professions. In looking at test scores across 67 countries and regions researchers found out that girls performed about as well or better than boys did on science in most countries, and in almost all countries, girls would have been capable of college-level science and math classes if they had enrolled in them.
	
	But when it comes to their relative strengths, in almost all the countries— boys’ best subject was science, and girls’ was reading. (That is, even if an average girl was as good as an average boy at science, she was still likely to be even better at reading.) Across all countries, 24 percent of girls had science as their best subject, 25 percent of girls’ strength was math, and 51 percent excelled in reading. For boys, the percentages were 38 for science, 42 for math, and 20 for reading. And the more gender-equal the country, as measured by the World Economic Forum’s Global Gender Gap Index, the larger this gap between boys and girls in having science as their best subject.
	
	\subsection{USA}
	
	“I remember walking into one of the classes at Stanford and just deciding not to take the class because I was one of only three women there, and I just felt so intimidated,”
	Catherina Xu – president Women’s Computer Science Society at Stanford University.
	
	Incidents like this are happening all across the country, and partly due to the lack of women in the field, there is now a shortage of computer science majors.
	
	The percentage of women in the field has been declining since the 1980s. The National Science Foundation found that in 1985, more than 35% of computer science majors were women. By 2014, that number had dropped down to just 18%.
	
	Laura Adolfie, the Florida STEM Chair for the American Association of University Women, believes part of the problem often starts in childhood. “When a child is born and you have a son or a daughter, they’re socialized by the parents and the grandparents. You tend to give a little girl a doll and a boy cars and things like that.” She said boys are socialized to tinker, which can start them on a path to engineering and computer science.
	
	So why aren’t more women working in computer science? 
	Three key factors are culture, the way women think and a lack of representation in the industry.
	
	
	\subsection{Example Africa}
	Moving to another continent... shows that the general problems are similar to the rest of the world. The cause may be a different one.
	For instance in Kenya, out of the top 100 best performing students only 17 were girls, and they were mostly from high-cost national secondary schools. 
	The statistic worsens as we go down to low-cost district secondary schools.
	So as we heard before the core of the problem lies also in the pre university education.
	The university of Nairobi did an evaluation where they found out that especially the didactic skills of the teachers and general quality of the classes drastically affect the interest of students in STEM topics. The worse the teacher is, the lower the interest is in STEM topics, affecting more female students because of their different approach of learning.
	
	
	\section{Conclusion}
	Gender equality and the empowerment of women and girls will make a crucial contribution to economic development of the world, and STEM education has a big role to play. As we have seen...